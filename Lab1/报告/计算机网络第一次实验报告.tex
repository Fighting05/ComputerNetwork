% !TeX program = xelatex
\documentclass[12pt, a4paper]{article}
\usepackage[UTF8]{ctex}
\usepackage{graphicx}
\usepackage{amsmath}
\usepackage{listings}
\usepackage{xcolor}
\usepackage{enumitem}
\usepackage{geometry}
\usepackage{hyperref}
\usepackage{float}
\usepackage{tcolorbox}
\usepackage{fancyhdr}
\usepackage{booktabs}
\usepackage{titlesec}
\usepackage{lmodern}

% 页面设置
\geometry{a4paper, margin=2.5cm, headheight=14pt}
\hypersetup{colorlinks=true, linkcolor=blue, urlcolor=blue}

% 颜色定义
\definecolor{codegreen}{rgb}{0.4,0.6,0.4}
\definecolor{codegray}{rgb}{0.5,0.5,0.5}
\definecolor{codepurple}{rgb}{0.58,0,0.82}
\definecolor{backcolour}{rgb}{0.95,0.95,0.95}
\definecolor{myblue}{RGB}{30, 144, 255}
\definecolor{mygray}{rgb}{0.95,0.95,0.95}

% 代码设置
\lstset{
    basicstyle=\small\ttfamily,
    keywordstyle=\color{blue}\bfseries,
    commentstyle=\color{codegreen},
    stringstyle=\color{red},
    numbers=left,
    numberstyle=\tiny\color{codegray},
    stepnumber=1,
    numbersep=5pt,
    backgroundcolor=\color{backcolour},
    tabsize=4,
    breaklines=true,
    breakatwhitespace=false,
    frame=single,
    rulecolor=\color{black},
    captionpos=b,
    linewidth=\textwidth
}

% 自定义命令
\newcommand{\cmd}[1]{\texttt{#1}}
\newcommand{\code}[1]{\texttt{\textcolor{blue}{#1}}}

% 标题格式
\titleformat{\section}
  {\normalfont\Large\bfseries\color{myblue}}
  {\thesection}{1em}{}

\titleformat{\subsection}
  {\normalfont\large\bfseries\color{myblue}}
  {\thesubsection}{1em}{}

\titleformat{\subsubsection}
  {\normalfont\normalsize\bfseries\color{myblue}}
  {\thesubsubsection}{1em}{}

% 页眉页脚设置
\pagestyle{fancy}
\fancyhf{}
\fancyhead[L]{\small 计算机网络实验报告}
\fancyhead[R]{\small 姓名:郭佳成 \quad 学号:2311990}
\fancyfoot[C]{\thepage}
\renewcommand{\headrulewidth}{0.4pt}
\renewcommand{\footrulewidth}{0.4pt}

% 提示框环境
\newtcolorbox{notebox}{
  colback=mygray,
  colframe=myblue,
  arc=4mm,
  boxrule=1pt,
  left=10pt,
  right=10pt,
  top=6pt,
  bottom=6pt
}

\title{\textbf{\LARGE 实验1:利用流式套接字编写聊天程序\\计算机网络第一次实验报告}}
\author{姓名:郭佳成 \quad 学号:2311990 \quad 专业:密码科学与技术 \\ \quad 代码:\url{https://github.com/Fighting05/ComputerNetwork}}
\date{\today}

\begin{document}

\maketitle

\section{实验要求}

\begin{enumerate}[itemsep=5pt]
  \item 设计聊天协议,并给出聊天协议的完整说明。
  \item 利用C或C++语言,使用基本的Socket函数进行程序编写,不允许使用CSocket等封装后的类。
  \item 程序应有基本的对话界面,但可以不是图形界面。程序应有正常的退出方式。
  \item 完成的程序应能支持英文和中文聊天。
  \item 采用多线程,支持多人聊天。
  \item 编写的程序应结构清晰,具有较好的可读性。
  \item 在实验中观察是否有数据包的丢失,提交程序源码、可执行代码和实验报告。
\end{enumerate}

\section{实验环境}

\begin{itemize}[itemsep=3pt]
  \item \textbf{操作系统}:Windows 11
  \item \textbf{开发语言}:C++11
  \item \textbf{编译器}:MinGW-w64 GCC
  \item \textbf{网络库}:Winsock2
  \item \textbf{GUI框架}:ImGui + GLFW + OpenGL3(图形界面版本)
  \item \textbf{服务器地址}:60.205.14.222:2059
  \item \textbf{本地测试}:127.0.0.1:1023
\end{itemize}

\section{实验原理}

\subsection{TCP流式套接字基础}

TCP(Transmission Control Protocol)是一种面向连接、可靠的传输层协议。流式套接字(Stream Socket)基于TCP协议,提供以下特性:

\subsection{Socket编程模型}



\subsubsection{服务器端流程}

服务器端的基本Socket编程流程包括以下几个步骤:首先调用\textbf{socket()}函数创建套接字,
然后使用\textbf{bind()}函数绑定IP地址和端口号,
接着调用\textbf{listen()}函数监听客户端连接请求,当有客户端连接时通过\textbf{accept()}
函数接受客户端连接,之后就可以使用\textbf{recv()/send()}函数进行数据的接收和发送,
最后在完成通信后调用\textbf{closesocket()}函数关闭套接字。

\subsubsection{客户端流程}

客户端的基本Socket编程流程包括以下几个步骤:首先调用\textbf{socket()}函数创建套接字,
然后使用\textbf{connect()}函数连接服务器,接着就可以通过\textbf{send()/recv()}函数进行数据的发送和接收,
最后在完成通信后调用\textbf{closesocket()}函数关闭套接字。

\subsection{多线程并发处理}

为了支持多人同时在线聊天,服务器需要为每个客户端创建独立的线程来处理消息收发。主要涉及:

\begin{itemize}[itemsep=3pt]
  \item \textbf{线程创建}:使用\cmd{std::thread}为每个客户端创建处理线程
  \item \textbf{线程同步}:使用\cmd{std::mutex}保护共享资源(客户端列表)
  \item \textbf{线程分离}:使用\cmd{detach()}让线程在后台独立运行
\end{itemize}

\section{聊天协议设计}

\subsection*{协议概述}

本聊天系统采用基于文本的应用层协议,使用TCP作为传输层协议。协议设计简洁高效,易于实现和扩展。



\subsection*{连接阶段}

客户端连接成功后的第一条消息为昵称注册:

\begin{lstlisting}[language=bash]
格式:<nickname>\n
示例:张三\n
说明:昵称不能包含换行符,建议长度不超过20个字符
\end{lstlisting}

\subsection*{普通消息(广播)}

客户端发送的普通消息会广播给所有在线用户:

\begin{lstlisting}[language=bash]
客户端发送格式:<message>\n
示例:大家好!\n

服务器转发格式:[<nickname>] <message>
示例:[张三] 大家好!
\end{lstlisting}

\subsection*{私聊消息}

使用特殊命令格式发送私聊消息:

\begin{lstlisting}[language=bash]
客户端发送格式:/msg <target_nickname> <message>\n
示例:/msg 李四 你好\n

接收方收到格式:[私聊] <sender_nickname>: <message>
示例:[私聊] 张三: 你好

发送方回显格式:<sender_nickname>:[私聊] TO<target_nickname>:<message>
示例:张三:[私聊] TO李四:你好
\end{lstlisting}

\subsection*{系统消息}

系统自动生成的通知消息:

\begin{lstlisting}[language=bash]
用户加入:[<nickname>] 加入了聊天室\n
用户离开:[<nickname>] 离开了聊天室\n
在线列表:当前在线用户: <user1>,<user2>,...\n
错误消息:用户 [<nickname>] 不在线\n
\end{lstlisting}


\subsection*{协议状态机}

\begin{figure}[H]
    \centering
    \begin{tcolorbox}[width=0.9\textwidth]
    \textbf{客户端状态转换:}
    \begin{enumerate}[itemsep=2pt]
      \item 未连接 → 发送昵称 → 已连接
      \item 已连接 → 发送消息/接收消息 → 已连接
      \item 已连接 → 断开连接 → 未连接
    \end{enumerate}
    
    \vspace{5pt}
    \textbf{服务器状态转换(针对每个客户端):}
    \begin{enumerate}[itemsep=2pt]
      \item 等待连接 → 接受连接 → 接收昵称
      \item 接收昵称 → 注册成功 → 通信状态
      \item 通信状态 → 接收/转发消息 → 通信状态
      \item 通信状态 → 检测断开 → 清理资源
    \end{enumerate}
    \end{tcolorbox}
    \caption{协议状态机}
\end{figure}

\section{程序设计与实现}

\subsection{系统架构}

本聊天系统采用经典的客户端/服务器(C/S)架构,包含以下三个主要组件:

\begin{enumerate}[itemsep=3pt]
  \item \textbf{服务器程序(server.cpp)}:负责管理客户端连接、消息转发和用户管理
  \item \textbf{控制台客户端(client.cpp)}:提供命令行界面的聊天客户端
  \item \textbf{图形界面客户端(client\_gui.cpp)}:提供基于ImGui的图形用户界面
\end{enumerate}

\subsection{服务器端实现}

\subsubsection{核心数据结构}

服务器使用以下数据结构管理客户端信息:

\begin{lstlisting}[language=c++]
// 存储所有客户端的Socket
vector<SOCKET> clients;

// Socket到昵称的映射
unordered_map<SOCKET, string> clientNames;

// 昵称到Socket的映射(用于私聊)
unordered_map<string, SOCKET> nameToSocket;

// 互斥锁,保护共享数据结构
mutex clientsMutex;
\end{lstlisting}

\subsubsection{主函数流程}

服务器主函数的核心逻辑如下:

\begin{lstlisting}[language=c++]
int main() {
    // 1. 初始化Winsock
    WSADATA wsaData;
    WSAStartup(MAKEWORD(2, 2), &wsaData);
    
    // 2. 创建Socket
    SOCKET serverSocket = socket(AF_INET, SOCK_STREAM, 0);
    
    // 3. 绑定地址和端口
    sockaddr_in addr;
    addr.sin_family = AF_INET;
    addr.sin_port = htons(1023);
    addr.sin_addr.s_addr = INADDR_ANY;
    bind(serverSocket, (sockaddr*)&addr, sizeof(addr));
    
    // 4. 开始监听
    listen(serverSocket, 5);
    
    // 5. 循环接受客户端连接
    while (true) {
        SOCKET clientSocket = accept(serverSocket, nullptr, nullptr);
        
        // 接收并注册客户端昵称
        char buf[1024];
        int n = recv(clientSocket, buf, sizeof(buf)-1, 0);
        buf[n] = '\0';
        string nickname(buf);
        if (!nickname.empty() && nickname.back() == '\n') {
            nickname.pop_back();
        }
        
        // 保存客户端信息
        {
            lock_guard<mutex> lock(clientsMutex);
            clientNames[clientSocket] = nickname;
            nameToSocket[nickname] = clientSocket;
            clients.push_back(clientSocket);
            
            // 广播加入消息
            string welcome = "[" + nickname + "] 加入了聊天室\n";
            for (SOCKET s : clients) {
                send(s, welcome.c_str(), welcome.size(), 0);
            }
        }
        
        // 为该客户端创建处理线程
        thread(handleClient, clientSocket).detach();
    }
    
    return 0;
}
\end{lstlisting}

\subsubsection{客户端处理函数}

每个客户端由独立线程处理,实现消息接收和转发:

\begin{lstlisting}[language=c++]
void handleClient(SOCKET clientSocket) {
    char buf[1024];
    while (true) {
        int n = recv(clientSocket, buf, sizeof(buf)-1, 0);
        if (n <= 0) {
            // 客户端断开连接
            lock_guard<mutex> lock(clientsMutex);
            string name = clientNames[clientSocket];
            clientNames.erase(clientSocket);
            nameToSocket.erase(name);
            clients.erase(remove(clients.begin(), clients.end(), 
                          clientSocket), clients.end());
            
            // 广播离开消息
            string goodbye = "[" + name + "] 离开了聊天室\n";
            for (SOCKET s : clients) {
                send(s, goodbye.c_str(), goodbye.size(), 0);
            }
            break;
        }
        
        buf[n] = '\0';
        string msg(buf);
        
        // 判断是否为私聊消息
        if (msg.length() >= 5 && msg.substr(0, 5) == "/msg ") {
            // 处理私聊逻辑
            string nickname = clientNames[clientSocket];
            size_t firstSpace = msg.find(' ', 5);
            
            if (firstSpace != string::npos) {
                string target = msg.substr(5, firstSpace - 5);
                string content = msg.substr(firstSpace + 1);
                string privateMsg = "[私聊] " + nickname + ": " + content;
                
                lock_guard<mutex> lock(clientsMutex);
                if (nameToSocket.count(target)) {
                    send(nameToSocket[target], privateMsg.c_str(), 
                         privateMsg.size(), 0);
                } else {
                    string err = "用户 [" + target + "] 不在线\n";
                    send(clientSocket, err.c_str(), err.size(), 0);
                }
            }
        } else {
            // 广播消息
            string nickname = clientNames[clientSocket];
            string broadcastMsg = "[" + nickname + "] " + msg;
            
            lock_guard<mutex> lock(clientsMutex);
            for (SOCKET s : clients) {
                send(s, broadcastMsg.c_str(), broadcastMsg.size(), 0);
            }
        }
    }
}
\end{lstlisting}

\subsection{控制台客户端实现}

\subsubsection{消息接收线程}

客户端使用独立线程接收服务器消息:

\begin{lstlisting}[language=c++]
void rcvFromServer(SOCKET clientSocket) {
    while(true) {
        string rcvBuf(1024, '\0');
        int bytesReceived = recv(clientSocket, &rcvBuf[0], 
                                rcvBuf.size(), 0);
        if (bytesReceived <= 0) {
            cout << "接收失败或连接关闭" << endl;
            return;
        }
        rcvBuf.resize(bytesReceived);
        cout << rcvBuf << endl;
    }
}
\end{lstlisting}

\subsubsection{消息发送线程}

另一个独立线程负责发送用户输入的消息:

\begin{lstlisting}[language=c++]
void sendServer(SOCKET clientSocket, const string& nickname) {
    while(true) {
        string buffer;
        cout << "请输入你要发送的消息(输入 /quit 退出): " << endl;
        getline(cin, buffer);
        
        if (buffer == "/quit") {
            cout << "你已经退出聊天" << endl;
            closesocket(clientSocket);
            return;
        }
        
        buffer += "\n";
        send(clientSocket, buffer.c_str(), buffer.size(), 0);
    }
}
\end{lstlisting}

\subsection{图形界面客户端实现}

图形界面客户端基于ImGui框架实现,提供了更友好的用户体验。

\subsubsection{全局数据结构}

\begin{lstlisting}[language=c++]
// 聊天记录
vector<string> g_chatHistory;

// 保护聊天记录的互斥锁
mutex g_chatMutex;

// 客户端Socket
SOCKET g_clientSocket = INVALID_SOCKET;

// 连接状态
bool g_connected = false;

// 用户昵称
string g_nickname;
\end{lstlisting}

\subsubsection{连接对话框}

程序启动时显示连接对话框,收集服务器信息和昵称:

\begin{lstlisting}[language=c++]
// 界面状态变量
char nicknameInput[128] = "";
char serverIP[64] = "60.205.14.222";
int serverPort = 2059;
bool showConnectDialog = true;

// 在主循环中渲染连接对话框
if (showConnectDialog && !g_connected) {
    // 创建全屏背景
    ImGui::SetNextWindowPos(ImVec2(0, 0), ImGuiCond_Always);
    ImGui::SetNextWindowSize(ImVec2(window_width, window_height), 
                            ImGuiCond_Always);
    
    ImGui::Begin("##ConnectionBackground", nullptr, 
                ImGuiWindowFlags_NoTitleBar | 
                ImGuiWindowFlags_NoResize);
    
    // 居中对话框
    ImVec2 dialogSize(1056, 768);
    ImVec2 dialogPos((window_width - dialogSize.x) * 0.5f, 
                    (window_height - dialogSize.y) * 0.5f);
    ImGui::SetCursorPos(dialogPos);
    
    ImGui::BeginChild("ConnectionDialog", dialogSize, true);
    
    // 标题
    ImGui::Text("连接到服务器");
    
    // 输入框
    ImGui::InputText("##ServerIP", serverIP, sizeof(serverIP));
    ImGui::InputInt("##Port", &serverPort);
    ImGui::InputText("##Nickname", nicknameInput, 
                    sizeof(nicknameInput));
    
    // 连接按钮
    if (ImGui::Button("连接", ImVec2(180, 45))) {
        if (strlen(nicknameInput) > 0) {
            if (connectToServer(serverIP, serverPort, 
                              string(nicknameInput))) {
                showConnectDialog = false;
            }
        }
    }
    
    ImGui::EndChild();
    ImGui::End();
}
\end{lstlisting}


\begin{figure}[H]
   \centering
   \includegraphics[width=0.8\textwidth]{pic/connect_dialog.png}
   \caption{连接对话框界面}
   \label{fig:connect_dialog}
\end{figure}

\subsubsection{主聊天窗口}

连接成功后显示主聊天窗口:

\begin{lstlisting}[language=c++]
if (g_connected || !showConnectDialog) {
    ImGui::SetNextWindowPos(ImVec2(0, 0), ImGuiCond_Always);
    ImGui::SetNextWindowSize(ImVec2(window_width, window_height), 
                            ImGuiCond_Always);
    
    ImGui::Begin("聊天室", nullptr, 
                ImGuiWindowFlags_NoResize | 
                ImGuiWindowFlags_MenuBar);
    
    // 菜单栏
    if (ImGui::BeginMenuBar()) {
        if (ImGui::BeginMenu("连接")) {
            if (ImGui::MenuItem("重新连接", nullptr, false, 
                              !g_connected)) {
                showConnectDialog = true;
            }
            if (ImGui::MenuItem("断开连接", nullptr, false, 
                              g_connected)) {
                closesocket(g_clientSocket);
                g_connected = false;
            }
            ImGui::EndMenu();
        }
        ImGui::EndMenuBar();
    }
    
    // 状态栏
    if (g_connected) {
        ImGui::TextColored(ImVec4(0.0f, 1.0f, 0.0f, 1.0f), 
                          "● 已连接");
    } else {
        ImGui::TextColored(ImVec4(1.0f, 0.0f, 0.0f, 1.0f), 
                          "● 未连接");
    }
    
    // 聊天记录区域
    ImGui::BeginChild("ChatHistory", ImVec2(0, chatHeight), true);
    {
        lock_guard<mutex> lock(g_chatMutex);
        for (const auto& msg : g_chatHistory) {
            ImGui::TextWrapped("%s", msg.c_str());
        }
    }
    if (autoScroll) {
        ImGui::SetScrollHereY(1.0f);
    }
    ImGui::EndChild();
    
    // 输入区域
    ImGui::Text("输入消息:");
    bool enterPressed = ImGui::InputText("##MessageInput", 
                        messageInput, sizeof(messageInput), 
                        ImGuiInputTextFlags_EnterReturnsTrue);
    
    ImGui::SameLine();
    bool sendClicked = ImGui::Button("发送", ImVec2(100, 38));
    
    // 发送消息
    if ((enterPressed || sendClicked) && strlen(messageInput) > 0) {
        if (g_connected) {
            string msg(messageInput);
            if (sendToServer(msg)) {
                messageInput[0] = '\0';
            }
        }
    }
    
    ImGui::End();
}
\end{lstlisting}

% 预留截图位置
\begin{figure}[H]
   \centering
   \includegraphics[width=0.9\textwidth]{pic/chat_window.png}
   \caption{主聊天窗口界面}
   \label{fig:chat_window}
\end{figure}


\section{编译与运行}

\subsection{运行步骤}

\begin{enumerate}[itemsep=3pt]
  \item \textbf{启动服务器}:在命令行运行\cmd{server.exe},服务器将在端口1023开始监听
  \item \textbf{启动客户端}:运行\cmd{client.exe}或\cmd{client\_gui.exe}
  \item \textbf{连接服务器}:输入服务器IP、端口和昵称,点击连接
  \item \textbf{开始聊天}:连接成功后即可发送消息
  \item \textbf{私聊功能}:使用\cmd{/msg 昵称 消息内容}格式发送私聊
  \item \textbf{退出程序}:控制台版输入\cmd{/quit},GUI版点击菜单中的退出
\end{enumerate}



\section{功能测试}

\subsection{基本功能测试}

\subsubsection{连接功能}

测试客户端能否成功连接到服务器:

\begin{itemize}[itemsep=3pt]
  \item \textbf{预期结果}:客户端显示"连接服务器成功",服务器显示"新客户端连接成功"
  \item \textbf{测试结果}:通过
\end{itemize}

% 预留截图位置
\begin{figure}[H]
   \centering
   \includegraphics[width=0.7\textwidth]{pic/server_runing.png}
   \caption{连接功能测试}
   \label{fig:connection_test}
\end{figure}

\subsubsection{昵称注册}

测试昵称是否正确注册和显示:

\begin{itemize}[itemsep=3pt]
  \item \textbf{预期结果}:服务器和所有客户端显示"[昵称] 加入了聊天室"
  \item \textbf{测试结果}:通过
\end{itemize}

\subsubsection{消息广播}

测试普通消息的广播功能:

\begin{itemize}[itemsep=3pt]
  \item \textbf{预期结果}:所有客户端都能收到格式为"[昵称] 消息内容"的消息
  \item \textbf{测试结果}: 通过
\end{itemize}


\subsubsection{私聊功能}

测试点对点私聊功能:

\begin{itemize}[itemsep=3pt]
  \item \textbf{测试目的}:验证私聊消息的正确传递
  \item \textbf{测试方法}:使用\cmd{/msg 目标昵称 消息}格式发送私聊
  \item \textbf{预期结果}:只有目标用户能收到私聊消息,格式为"[私聊] 发送者: 消息"
  \item \textbf{测试结果}:✓ 通过
\end{itemize}

% 预留截图位置
\begin{figure}[H]
   \centering
   \includegraphics[width=0.9\textwidth]{pic/private_chat_test.png}
   \caption{私聊功能测试}
   \label{fig:private_chat_test}
\end{figure}

\subsection{异常情况测试}

\subsubsection{网络断开测试}

测试客户端异常断开时的处理:

\begin{itemize}[itemsep=3pt]
  \item \textbf{测试方法}:强制关闭客户端进程或断开网络连接
  \item \textbf{预期结果}:
  \begin{enumerate}[itemsep=2pt]
    \item 服务器检测到连接断开
    \item 服务器清理该客户端的资源
    \item 向其他客户端广播"[昵称] 离开了聊天室"
    \item 不影响其他客户端的正常通信
  \end{enumerate}
  \item \textbf{测试结果}:✓ 通过
\end{itemize}

通过在\cmd{recv()}函数检测返回值为0或负数来判断连接断开,并及时清理资源。

\subsubsection{无效私聊目标测试}

测试向不存在的用户发送私聊消息:

\begin{itemize}[itemsep=3pt]
  \item \textbf{测试方法}:使用\cmd{/msg 不存在的昵称 消息}发送私聊
  \item \textbf{预期结果}:发送者收到"用户 [昵称] 不在线"的提示
  \item \textbf{测试结果}:✓ 通过
\end{itemize}

\subsubsection{重复昵称测试}

由于当前实现未限制重复昵称,多个用户可以使用相同的昵称。这是系统的一个可改进点:

\begin{itemize}[itemsep=3pt]
  \item \textbf{当前行为}:允许重复昵称,使用后连接的用户覆盖之前的映射
  \item \textbf{建议改进}:在用户注册时检查昵称是否已存在,拒绝重复昵称
\end{itemize}

\section{数据包丢失分析}

本实验要求观察程序运行过程中是否存在数据包丢失。由于聊天程序基于 TCP 流式套接字实现,而 TCP 协议本身具有可靠性保障机制(如确认应答、超时重传、顺序控制等),在正常网络环境下理论上不会发生底层数据包丢失。因此,测试的重点在于验证应用程序是否能正确发送和接收完整消息,避免因编码错误、缓冲区处理不当或粘包问题导致应用层“逻辑丢包”。

为直观验证消息传输的完整性,我们使用 Wireshark 在本地回环接口上对通信过程进行抓包分析。测试时,服务器运行于本地 1023 端口,客户端连接至 \texttt{127.0.0.1:1023}。在 Wireshark 中选择 Npcap Loopback Adapter 接口,并设置过滤条件 \texttt{tcp.port == 1023},以仅捕获聊天程序的通信流量。

客户端发送英文消息 \texttt{hello} 后,Wireshark 成功捕获到对应的 TCP 数据包,如图~\ref{fig:wireshark_hello} 所示。该数据包源端口为客户端临时端口(如 3032),目的端口为 1023,TCP 标志位为 [PSH, ACK],表明其携带有效应用层数据。在数据负载部分,十六进制内容显示为 \texttt{68 65 6c 6c 6f 0a},恰好对应 ASCII 字符串 \texttt{hello},总长度为 6 字节,与程序发送内容完全一致。同时,服务器与其他客户端均能正常显示该消息,未出现截断、乱码或缺失现象。

\begin{figure}[H]
   \centering
   \includegraphics[width=0.95\textwidth]{pic/wireshark_hello.png}
   \caption{Wireshark 抓包结果:客户端发送 "hello" 消息}
   \label{fig:wireshark_hello}
\end{figure}

上述结果表明,程序在发送消息时调用 \texttt{send()} 成功,TCP 协议栈完整传输了数据,接收端也能正确解析并输出。在整个测试过程中,未观察到任何数据包丢失或内容损坏的情况。尽管当前接收逻辑未显式处理粘包(依赖单次 \texttt{recv} 读取完整消息),但在低频、短消息场景下仍能正常工作。若未来扩展至高并发或长消息场景,建议引入基于换行符的消息帧解析机制以进一步提升鲁棒性。

综上,通过 Wireshark 抓包验证,本聊天程序在常规使用条件下能够可靠传输消息,满足实验对“观察数据包丢失”的要求。


\section{总结}

这次实验让我第一次完整地用原始 Socket API 实现了一个多人聊天程序。从服务器监听、客户端连接,到多线程处理消息、支持中英文,每一步都遇到了问题,也都在调试中解决了。
虽然代码还有改进空间(比如粘包处理、昵称去重),但基本功能都跑通了,也能稳定聊天。整个过程加深了我对 TCP 流式套接字和多线程并发的理解,也让我意识到,写网络程序不能只依赖协议的“可靠性”,应用层的细节同样关键。这次实验收获很大,也为后续更复杂的网络编程打下了基础。
\section*{编译和运行说明}

\subsection*{环境要求}

\begin{itemize}[itemsep=3pt]
  \item Windows 7或更高版本
  \item MinGW-w64 GCC编译器(支持C++11)
  \item 对于GUI版本,需要GLFW和OpenGL支持
\end{itemize}

\subsection*{编译步骤}

\textbf{1. 编译服务器:}
\begin{lstlisting}[language=bash]
g++ -std=c++11 server.cpp -o server.exe -lws2_32 -static -static-libgcc -static-libstdc++
\end{lstlisting}

\textbf{2. 编译控制台客户端:}
\begin{lstlisting}[language=bash]
g++ -std=c++11 client.cpp -o client.exe -lws2_32 -static -static-libgcc -static-libstdc++
\end{lstlisting}

\textbf{3. 编译GUI客户端:}
\begin{lstlisting}[language=bash]
g++ -std=c++11 client_gui.cpp imgui/*.cpp imgui/backends/imgui_impl_glfw.cpp imgui/backends/imgui_impl_opengl3.cpp -o client_gui.exe -I./imgui -I./imgui/backends -lglfw3 -lopengl32 -lgdi32 -lws2_32 -static -static-libgcc -static-libstdc++
\end{lstlisting}

\subsection*{运行步骤}

\begin{enumerate}[itemsep=3pt]
  \item 首先运行\cmd{server.exe}启动服务器
  \item 运行一个或多个\cmd{client.exe}或\cmd{client\_gui.exe}启动客户端
  \item 在客户端中输入服务器IP、端口和昵称
  \item 连接成功后即可开始聊天
\end{enumerate}


\end{document}